\documentclass[a4paper,12pt]{article}
\usepackage{geometry}
\geometry{left=3cm, right=2cm, top=3cm, bottom=2cm}
\usepackage[portuguese,brazil]{babel}
\usepackage[utf8]{inputenc}
\usepackage[T1]{fontenc}
\usepackage{lmodern}
\usepackage{amsmath}
\usepackage{multicol}
\usepackage{amsfonts}
\usepackage{amssymb}
\usepackage{pgfplots}
\usepackage{calc}
\usepackage{listings}

\usepackage{xcolor} % necessário para cores

\definecolor{codeback}{RGB}{245,245,248}
\definecolor{string}{RGB}{0,120,0}
\definecolor{keyword}{RGB}{0,0,180}
\definecolor{comment}{RGB}{120,120,120}
\definecolor{number}{RGB}{180,0,0}

\lstset{
    backgroundcolor=\color{codeback},
    basicstyle=\ttfamily\small,
    keywordstyle=\color{keyword}\bfseries,
    stringstyle=\color{string},
    commentstyle=\color{comment}\itshape,
    numberstyle=\tiny\color{gray},
    numbers=left,
    numbersep=10pt,
    stepnumber=1,
    showspaces=false,
    showstringspaces=false,
    tabsize=4,
    frame=single,
    frameround=tttt,
    rulecolor=\color{black!30},
    breaklines=true,
    breakatwhitespace=true,
    captionpos=b,
    abovecaptionskip=10pt,
    belowcaptionskip=10pt,
    xleftmargin=15pt,
    xrightmargin=10pt,
    columns=flexible,
    keepspaces=true,
    escapeinside={(*@}{@*)} % permite LaTeX dentro do código
}

% Estilo específico para Python
\lstdefinestyle{python}{
    language=Python,
    morekeywords={*,import,as,from,None,True,False,self,np},
    emph={array,arange,sum,mean,std,max,min,dot,cross,sin,exp,log},
    emphstyle=\color{blue}\bfseries
}

% Estilo específico para C#
\lstdefinestyle{csharp}{
    language=[Sharp]C,
    morekeywords={using,var,new,public,private,static,void,double,float,int,bool},
    emph={Array,List,Enumerable,Range,Select,Sum,Average},
    emphstyle=\color{blue}\bfseries
}

\pgfplotsset{compat=1.18}
\numberwithin{equation}{section}

\title{Revisão e Notas sobre Vetores}
\author{Pedro Rupf Pereira Viana}
\date{\today}

\begin{document}

\maketitle
\newpage

\section{Introdução e Objetivos}
Trata-se de uma revisão teórica e expositiva sobre vetores e suas definições fundamentais no plano cartesiano, e suas aplicações na matemática, física e na programação.

\section{Desenvolvimento}

Consideremos o plano cartesiano em um sistema de coordenadas em $\mathbb{R}^{2}$, formado por um par de retas ortogonais. Fixada uma unidade de comprimento, 
qualquer ponto $P$ do plano pode ser identificado pelo par ordenado $(a,b) \in \mathbb{R}^2$, onde $a$ é a abscissa e $b$ é a ordenada.

\begin{figure}[h]
\centering
\begin{tikzpicture}
\begin{axis}[
    axis lines = middle,
    xlabel = {$x$},
    ylabel = {$y$},
    xmin=-1, xmax=2,
    ymin=-1, ymax=2,
    xtick={-1,0,1,2},
    ytick={-1,0,1,2},
    grid=both,
    minor tick num=1,
    grid style={dashed, gray!40},
    width=6cm,
    height=6cm,
]

\draw[->, thick] (axis cs:0,0) -- (axis cs:1.4,0);
\draw[->, thick] (axis cs:0,0) -- (axis cs:0,1.4);

\addplot[mark=*, mark size=3.5pt, mark options={fill=black, draw=black}]
    coordinates {(1,1)}
    node[above right, font=\small] {$P(a,b)$};

\draw[dashed, gray!70] (1,0) -- (1,1);
\draw[dashed, gray!70] (0,1) -- (1,1);

\end{axis}
\end{tikzpicture}
\caption{Representação do ponto $P(a,b)$ no primeiro quadrante do plano cartesiano.}
\label{fig:ponto_P}
\end{figure}

Desta forma, dados dois pontos $P$ e $Q$ do plano, adotando $Q(0,0)$ (isto é, a origem do plano cartesiano), podemos considerar que o segmento de reta 
$\overrightarrow{QP}$, com ponto inicial $Q$ e ponto final $P$. Note que embora como conjunto de pontos os segmentos $\overrightarrow{QP}$ e $\overrightarrow{PQ}$ 
sejam iguais, como segmentos orientados eles são distintos, onde chamamos estes vetores de segmentos opostos.

Passemos a considerar, à partir de agora, apenas segmentos orientados com ponto inicial na origem, denominados \textit{vetores no plano}. É importante notar que 
vetores no plano são determinados exclusivamente pelo seu ponto final, pois o ponto inicial é fixo na origem. Assim, para cada ponto no plano $P(a,b)$, está associado
um único vetor $\textbf{v} = \overrightarrow{OP}$. Usando esta correspondência entre pontos e vetores, podemos representar o vetor $\textbf{v} = \overrightarrow{OP}$ 
pela identificação $\textbf{v} = (1,3)$, ou ainda pela notação matriz-coluna $\textbf{v} = \begin{bmatrix} 1 \\ 3 \end{bmatrix}$. Observe que, deste modo, à origem 
do plano, ficará associado um vetor que têm os pontos inicial e final coincidentes com esta. Denominaremos este vetor (que, na verdade, é apenas um ponto) de 
\textit{vetor nulo}, sendo representado por $(0,0)$.

O oposto de um vetor $\textbf{v} = \overrightarrow{OP}$ é o vetor $\textbf{w} = \overrightarrow{OQ}$, que possui o mesmo comprimento que $\textbf{v}$, porém com 
direção oposta. Em termos de coordenadas, se $\textbf{v} = (a,b)$, então $\textbf{w} = (-a,-b)$ e, por essa razão, denota-se que $\textbf{w} = -\textbf{v}$.

\subsection{Operações com vetores no plano}

\subsubsection{Multiplicação de um vetor por um número}

Multiplicar um vetor $\textbf{v}$ por um número real $k > 0$ é considerar um novo vetor $k\textbf{w} = kv$, que possui a mesma direção de $\textbf{v}$ e têm como 
comprimento $k$ vezes o comprimento de $\textbf{v}$. Se $k < 0$, o vetor $\textbf{w} = kv$ terá direção oposta à de $\textbf{v}$ e comprimento $|k|$ vezes o 
comprimento de $\textbf{v}$. Se $k = 0$, o vetor resultante $\textbf{w} = kv$ será o vetor nulo.

\begin{figure}[h]
\centering
\begin{tikzpicture}
\begin{axis}[
    axis lines = middle,
    xlabel = {$x$},
    ylabel = {$y$},
    xmin=-1, xmax=3,
    ymin=-1, ymax=3,
    xtick={-1,0,1,2},
    ytick={-1,0,1,2},
    grid=both,
    minor tick num=1,
    grid style={dashed, gray!40},
    width=6cm,
    height=6cm,
]

% Desenhando os vetores
\draw[->, thick, red] (axis cs:0,0) -- (axis cs:2,2);  % Vetor w = 2v
\draw[->, thick, black] (axis cs:0,0) -- (axis cs:1,1);  % Vetor v
\draw[->, thick, blue] (axis cs:0,0) -- (axis cs:-0.5,-0.5);  % Vetor t = -0.5v

% Linhas de projeção do ponto
\draw[dashed, gray!70] (1,0) -- (1,1);
\draw[dashed, gray!70] (0,1) -- (1,1);

\end{axis}
\end{tikzpicture}
\caption{Representação dos vetores \( \mathbf{v} \), \( \mathbf{w} = 2\mathbf{v} \), e \( \mathbf{t} = -0.5\mathbf{v} \) no plano cartesiano.}
\label{fig:vetores}
\end{figure}

\subsubsection{Adição de dois vetores}

Para introduzir a soma de dois vetores, consideremos um exemplo de duas forças atuando sobre um corpo, representadas pelos vetores $\mathbf{F}_1$ e $\mathbf{F}_2$, 
conforme podemos ver na Figura 3 abaixo:

\begin{figure}[h]
\centering
\begin{tikzpicture}[
    thick,
    >=stealth,
    force/.style={->, red, very thick},
    res/.style={->, green!60!black, very thick}
  ]

% --- Superfície (chão) ---
\draw[line width=3pt] (-4,0) -- (4,0);

% --- Bloco: retângulo com sombra ---
\fill[gray!40, opacity=0.4] (-1.4,0.1) rectangle (1.4,1.6);
\draw[fill=cyan!30, draw=black, line width=1.2pt]
      (-1.4,0.1) rectangle (1.4,1.6);

% hachuras no chão (apoio)
\foreach \x in {-3.8,-3.3,...,3.8}
   \draw[gray] (\x,0) -- ++(0,-0.2);

% --- Centro do bloco (exatamente no meio) ---
\coordinate (O) at (0, 0.85);

% --- Forças ---
% F1: paralela ao plano (horizontal)
\draw[force] (O) -- ++(0:3.5cm) node[right] {$\mathbf{F_1}$};

% F2: inclinada, formando α com F1 (ex: 55°)
\draw[force] (O) -- ++(55:3.8cm) node[above left=3pt] {$\mathbf{F_2}$};

% --- Ângulo α entre F1 e F2 (sempre < 90°) ---
\draw[blue, thick]
      (O) ++(0:1.3cm) arc[start angle=0, end angle=55, radius=1.3cm];

% --- Resultante (regra do paralelogramo) ---
\draw[res, dashed] (O) -- ++(27:4.8cm) node[above right] {$\mathbf{F_R}$};

% --- Legenda ---
\node[font=\small] at (0,-0.7) {superfície};

\end{tikzpicture}
\caption{Bloco sobre uma superfície sujeito a duas forças: $\mathbf{F_1}$ paralela ao plano e $\mathbf{F_2}$ formando ângulo $\alpha < 90^\circ$ com $\mathbf{F_1}$. 
A força resultante $\mathbf{F_R}$ é obtida pela regra do paralelogramo.}
\label{fig:forcas_bloco_retangulo}
\end{figure}

Uma força que atua num ponto pode ser representada por um vetor, de comprimento igual à intensidade da força, com a mesma direção e sentido desta força. Supondo agora 
que as duas forças $\mathbf{F}_1$ e $\mathbf{F}_2$, representadas na FIgura 3, atuem simultaneamente sobre o corpo apresentado nesta mesma figura. Podemos representar 
o resultado destas duas forças por uma única força $\mathbf{F_R}$?

Ora, podemos representar que a força resultante $\mathbf{F_R}$ é obtida pelo vetor diagonal do paralelogramo construído a partir dos vetores $\mathbf{F_1}$ e 
$\mathbf{F_2}$, chamando esta operação de soma de vetores, onde escrevemos $\mathbf{F_R} = \mathbf{F_1} + \mathbf{F_2}$. De acordo com a Figura 4 abaixo, temos que 
$\vec{R} = \vec{v} + \vec{u}$.

\begin{figure}[h]
		\centering
		\includegraphics[width=7cm]{image-5.png}
		\caption{Regra do paralelogramo aplicada à soma de vetores.}
\end{figure}

\subsubsection{Vetores no espaço}

Da mesma forma que é definido vetores em um espaço em $\mathbb{R}^{2}$, podemos definir vetores em um espaço tridimensional, ou seja, em $\mathbb{R}^{3}$. Neste caso, 
um ponto $P$ do espaço é identificado por um triplo ordenado $(a,b,c)$, onde temos um sistema de coordenadas formado por três retas ortogonais entre si. Assim, um 
vetor são dados por um segmento orientado com ponto inicial na origem e ponto final em $P(a,b,c)$, representado por $\textbf{v} = (a,b,c)$.

\begin{figure}[h]
		\centering
		\includegraphics[width=8cm]{002.png}
		\caption{Vetor genérico $\textbf{v} = (2,4,3)$, representado no espaço em $\mathbb{R}^{3}$.}
\end{figure}

\subsubsection{Operações com vetores no espaço}

A soma de dois vetores, e o produto de um vetor por um número real $k$ em $\mathbb{R}^{3}$ também são da mesma forma que no plano. Isto é, se 
$\mathbf{u} = ({x}_1 + {x}_2 + {x}_3)$ e $\mathbf{w} = ({y}_1 + {y}_2 + {y}_3)$, então a soma dos vetores é dada por:

\begin{eqnarray}
	\mathbf{u} + \mathbf{w} = (x_1 + y_1, x_2 + y_2, x_3 + y_3)
\end{eqnarray}

\begin{eqnarray}
	k\mathbf{u} = (kx_1, kx_2, kx_3)
\end{eqnarray}

Por exemplo, se $\mathbf{u} = (2,-3,5)$ e $\mathbf{v} = (1,2,0)$, então $\mathbf{u} + \mathbf{w} = (3,-1,5)$, e $2\mathbf{u} = (4,-6,10)$.

Como já observamos no caso do plano, estas operações correspondem exatamente às respectivas operações das matrizes linha que representam os vetores, e gozam de uma 
série de propriedades decorrentes daquelas relativas às operações com números reais.

\subsubsection{Propriedades vetoriais}

\begin{multicols}{2}
    \begin{itemize}
        \item $(\mathbf{u} + \mathbf{v}) + \mathbf{w} = \mathbf{u} + (\mathbf{v} + \mathbf{w})$
        \item $\mathbf{u} + \mathbf{v} = \mathbf{v} + \mathbf{u}$
        \item Existe $\mathbf{0}$ $\in$ $V$ tal que $\mathbf{u} + \mathbf{0} = \mathbf{u}$
        \item Existe $-\mathbf{u}$ $\in$ $V$ tal que $\mathbf{u} + (-\mathbf{u}) = \mathbf{0}$
    \end{itemize}
    \columnbreak
    \begin{itemize}
        \item $a(\mathbf{u} + \mathbf{v}) = a\mathbf{u} + a\mathbf{v}$
        \item $(a + b)\mathbf{v} = a\mathbf{v} + b\mathbf{v}$
        \item $(ab)\mathbf{v} = a(b\mathbf{v})$
        \item $1\mathbf{u} = \mathbf{u}$
    \end{itemize}
\end{multicols}

\section{Ok...e onde isso se aplica na programação?}

Um \textbf{vetor} é uma estrutura de dados que armazena uma coleção \textbf{ordenada} de elementos do mesmo tipo (ou tipos compatíveis), acessíveis por meio de um 
\textbf{índice numérico}. Em termos matemáticos, um vetor é uma grandeza com magnitude e direção; na programação, ele representa uma \textbf{sequência indexada} de 
valores.

Em linguagens de baixo nível como C ou C++, vetores são implementados como \textbf{arrays} estáticos (tamanho fixo na memória). Em C$\#$ e em Python, o conceito de vetor 
é mais flexível e geralmente representado por \textbf{Lists} (ou Arrays, no caso do C$\#$), ou também por Lists e pela biblioteca \textbf{NumPy} (que oferece verdadeiros 
arrays n-dimensionais otimizados).

\subsection{}

\begin{table}[ht]
\centering
\begin{tabular}{|c|c|c|}
\hline
Cabeçalho 1 & Cabeçalho 2 & Cabeçalho 3 \\ \hline
Valor 1 & Valor 2 & Valor 3 \\ \hline
Valor 4 & Valor 5 & Valor 6 \\ \hline
Valor 7 & Valor 8 & Valor 9 \\ \hline
Valor 10 & Valor 11 & Valor 12 \\ \hline
Valor 13 & Valor 14 & Valor 15 \\ \hline
Valor 16 & Valor 17 & Valor 18 \\ \hline
\end{tabular}
\caption{Exemplo de Tabela 6x3}
\end{table}


\begin{lstlisting}[style=python, caption={Uso de list como vetor em Python}, label={lst:lista_vetor}]
vetor_lista = [10, 20, 30, 40, 50]

# Acesso por indice
print(f"Primeiro elemento: {vetor_lista[0]}")
print(f"Ultimo elemento: {vetor_lista[-1]}")

# Modificacao
vetor_lista[2] = 999
print(f"Vetor modificado: {vetor_lista}")

# Soma manual
soma = sum(vetor_lista)
print(f"Soma (funcao sum): {soma}")

# Multiplicacao por escalar (list comprehension)
vetor_dobrado = [x * 2 for x in vetor_lista]
print(f"Vetor dobrado: {vetor_dobrado}")
\end{lstlisting}

\end{document}