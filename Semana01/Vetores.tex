\documentclass[a4paper,12pt]{article}
\usepackage{geometry}
\geometry{left=3cm, right=2cm, top=3cm, bottom=2cm}
\usepackage[portuguese,brazil]{babel}
\usepackage[utf8]{inputenc}
\usepackage[T1]{fontenc}
\usepackage{lmodern}
\usepackage{amsmath}
\usepackage{amsfonts}
\usepackage{amssymb}
\usepackage{pgfplots}
\pgfplotsset{compat=1.18}
\numberwithin{equation}{section}

\title{Revisão e Notas sobre Vetores}
\author{Pedro Rupf Pereira Viana}
\date{\today}

\begin{document}

\maketitle
\newpage

\section{Introdução e Objetivos}
Trata-se de uma revisão teórica e expositiva sobre vetores e suas definições fundamentais no plano cartesiano, e suas aplicações na matemática, física e na programação.

\section{Desenvolvimento}

Consideremos o plano cartesiano em um sistema de coordenadas em $\mathbb{R}^{2}$, formado por um par de retas ortogonais. Fixada uma unidade de comprimento, 
qualquer ponto $P$ do plano pode ser identificado pelo par ordenado $(a,b) \in \mathbb{R}^2$, onde $a$ é a abscissa e $b$ é a ordenada.

\begin{figure}[h]
\centering
\begin{tikzpicture}
\begin{axis}[
    axis lines = middle,
    xlabel = {$x$},
    ylabel = {$y$},
    xmin=-1, xmax=2,
    ymin=-1, ymax=2,
    xtick={-1,0,1,2},
    ytick={-1,0,1,2},
    grid=both,
    minor tick num=1,
    grid style={dashed, gray!40},
    width=6cm,
    height=6cm,
]

\draw[->, thick] (axis cs:0,0) -- (axis cs:1.4,0);
\draw[->, thick] (axis cs:0,0) -- (axis cs:0,1.4);

\addplot[mark=*, mark size=3.5pt, mark options={fill=black, draw=black}]
    coordinates {(1,1)}
    node[above right, font=\small] {$P(a,b)$};

\draw[dashed, gray!70] (1,0) -- (1,1);
\draw[dashed, gray!70] (0,1) -- (1,1);

\end{axis}
\end{tikzpicture}
\caption{Representação do ponto $P(a,b)$ no primeiro quadrante do plano cartesiano.}
\label{fig:ponto_P}
\end{figure}

Desta forma, dados dois pontos $P$ e $Q$ do plano, adotando $Q(0,0)$ (isto é, a origem do plano cartesiano), podemos considerar que o segmento de reta 
$\overrightarrow{QP}$, com ponto inicial $Q$ e ponto final $P$. Note que embora como conjunto de pontos os segmentos $\overrightarrow{QP}$ e $\overrightarrow{PQ}$ 
sejam iguais, como segmentos orientados eles são distintos, onde chamamos estes vetores de segmentos opostos.

Passemos a considerar, à partir de agora, apenas segmentos orientados com ponto inicial na origem, denominados \textit{vetores no plano}. É importante notar que 
vetores no plano são determinados exclusivamente pelo seu ponto final, pois o ponto inicial é fixo na origem. Assim, para cada ponto no plano $P(a,b)$, está associado
um único vetor $\textbf{v} = \overrightarrow{OP}$

\subsection{Exemplos Matemáticos}

Aqui estão alguns exemplos clássicos de expressões matemáticas:

\begin{eqnarray}
	ax^2 + bx + c = 0
\end{eqnarray}

\begin{eqnarray}
	\rho_m\left(\vec{v}\cdot\vec{\nabla}\right)\vec{v}=-\vec{\nabla}P-\vec{\nabla}\cdot\overleftrightarrow{\Pi}
\end{eqnarray}

\subsection{Plano Cartesiano com Vetores}

Nesta subseção, exploraremos a representação geométrica de vetores a partir da origem até pontos do plano, como $\overrightarrow{OP}$, onde $P(a,b)$.

% (Você pode adicionar mais gráficos aqui depois)

\section{Conclusão}

Esta revisão consolidou os conceitos fundamentais de vetores no plano cartesiano, incluindo sua representação por coordenadas, operações básicas e visualização geométrica. O uso de ferramentas como o \LaTeX\ e o \texttt{pgfplots} permite uma apresentação clara, precisa e esteticamente agradável do conteúdo matemático.

\end{document}