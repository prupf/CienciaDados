\documentclass[a4paper,12pt]{article}  % Define o tipo de documento (artigo), papel A4 e fonte tamanho 12pt

\usepackage[utf8]{inputenc}  % Define a codificação do documento para UTF-8
\usepackage{amsmath}         % Pacote para matemática
\usepackage{amsfonts}        % Pacote para fontes matemáticas
\usepackage{amssymb}         % Pacote para símbolos matemáticos

\title{Exemplo de Documento LaTeX com Título e Estrutura de Tópicos}  % Título do documento
\author{Pedro Rupf Pereira Viana}            % Autor
\date{\today}                % Data (data de compilação)

\begin{document}  % Início do documento

\maketitle  % Cria o título, autor e data

\begin{abstract}  % Resumo do documento
    Este é um exemplo de documento LaTeX, que contém título, divisões e subdivisões por tópicos. 
    A estrutura básica de um documento acadêmico é demonstrada, incluindo seções e subseções.
\end{abstract}

\tableofcontents  % Gera o índice automaticamente com as seções e subseções

\newpage  % Quebra de página

\section{Introdução}  % Seção principal
Aqui começa o conteúdo da introdução. Neste documento, iremos demonstrar como organizar um artigo em LaTeX com diferentes divisões e subdivisões.

\subsection{Objetivo}  % Sub-seção
O objetivo deste exemplo é fornecer uma estrutura básica para a criação de documentos em LaTeX com títulos, seções e subseções.

\subsection{Estrutura do Documento}  % Sub-seção
O documento será organizado da seguinte forma:
\begin{itemize}
    \item Seção de Introdução
    \item Seção de Desenvolvimento
    \item Seção de Conclusão
\end{itemize}

\section{Desenvolvimento}  % Seção principal
Nesta seção, apresentamos os detalhes do desenvolvimento do documento. LaTeX permite incluir facilmente matemáticas, gráficos, e tabelas.

\subsection{Subseção de Matemática}  % Sub-seção
Aqui estão alguns exemplos de expressões matemáticas:

Equação quadrática: 
\[
ax^2 + bx + c = 0
\]

Teorema de Pitágoras:
\[
a^2 + b^2 = c^2
\]

\subsection{Subseção de Gráficos}  % Sub-seção
Você pode incluir gráficos em LaTeX utilizando pacotes como `graphicx`. Por exemplo:

\begin{figure}[h]
    \centering
    % Aqui você incluiria um arquivo de imagem, como um gráfico
    \includegraphics[width=0.5\textwidth]{exemplo.jpg}
    \caption{Exemplo de Gráfico}
    \label{fig:grafico}
\end{figure}

\section{Conclusão}  % Seção principal
Este é o fechamento do documento. Aqui discutimos as conclusões sobre o tema abordado.

\subsection{Considerações Finais}  % Sub-seção
LaTeX é uma ferramenta poderosa para a criação de documentos acadêmicos. Ele oferece um controle preciso sobre a formatação e permite incluir fórmulas matemáticas complexas, gráficos e tabelas com facilidade.

\subsection{Sugestões para Trabalhos Futuros}  % Sub-seção
Futuramente, seria interessante explorar o uso de LaTeX para criar apresentações (com o pacote Beamer), além de integração com bibliotecas externas.

\end{document}  % Fim do documento
